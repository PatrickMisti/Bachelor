% !TeX program = pdflatex
\documentclass[a4paper,12pt,oneside]{scrreprt} % KOMA-Script für Abschlussarbeiten

% ====== Pakete ======
\usepackage[utf8]{inputenc}
\usepackage[T1]{fontenc}
\usepackage[ngerman]{babel}        % Deutsche Sprache
\usepackage{csquotes}              % Korrekte Anführungszeichen
\usepackage{graphicx}              % Bilder einbinden
\usepackage{float}                 % Genaue Bildpositionierung
\usepackage{amsmath,amssymb}       % Mathematische Formeln
\usepackage{hyperref}              % Links und PDF-Metadaten
\usepackage{geometry}              % Seitenränder
\usepackage{setspace}              % Zeilenabstand
\usepackage{booktabs}              % Schönere Tabellen
\usepackage{caption}               % Bessere Bildunterschriften
\usepackage{listings}              % Quellcode
\usepackage[
  backend=biber,
  style=authoryear,
  citestyle=authoryear,
  sorting=nyt,
  maxcitenames=2,   % ab 3 Autoren -> et al.
  maxbibnames=99,   % Bibliographie nicht unnötig kürzen
  giveninits=true,  % Initialen statt volle Vornamen
  uniquename=false  % KEIN Erzwingen von ausgeschriebenen Vornamen
]{biblatex}              % Literaturverwaltung (modern)

\addbibresource{literatur.bib}

\renewcommand*{\mkbibparens}[1]{[#1]}
% ====== Layout ======
\geometry{a4paper, left=3cm, right=2.5cm, top=2.5cm, bottom=2.5cm}
\setstretch{1.5} % 1.5-zeilig
\setlength{\parindent}{0pt}
\setlength{\parskip}{6pt}
\hypersetup{
    colorlinks=true,
    linkcolor=blue,
    urlcolor=blue,
    citecolor=black,
    pdftitle={Bachelorarbeit},
    pdfauthor={Dein Name},
}

% ====== Dokument ======
\begin{document}

\begin{titlepage}
\centering
{\Large Fachhochschule Hagenberg \\[1cm]}
{\Huge \textbf{Konzeption und Evaluierung eines verteilten, aktorenbasierten Stream-Verarbeitungssystems mit
Akka.NET}}\\[1cm]
{\large Bachelorarbeit im Studiengang Software Entwicklung}\\[2cm]
\textbf{Autor:} Patrick Mistlberger\\
\textbf{Matrikelnummer:} 123456\\[0.5cm]
\textbf{Betreuer:} Prof. Dr. Mustermann\\[2cm]
\today
\end{titlepage}

\tableofcontents
\clearpage

\chapter{Einleitung}
\section{Motivation}
Diese Bachelorarbeit beschäftigt sich mit dem Thema \emph{...}.
Ziel ist es, die Problemstellung zu analysieren, Methoden zu vergleichen
und ein eigenes Konzept zu entwickeln.

Die Motivation ergibt sich aus ...
Na servus

\section{Aufgabenstellung}
\section{Ziel der Arbeit}
\section{Vorgehensweise}
\chapter{Grundlagen}

\section{Einführung in Reactive Systems und Aktorenmodell}

\section{Akka.Net Framework}

\subsection{Cluster}
\subsection{Aktoren}
was passiert bei ausfall einer nachricht messages erklären?
\subsection{Sharding}
shard in der shardregio was ist das erklären?
\subsection{Singleton}
\subsection{Verteiltes Pub/Sub}
was passiert bei ausfall einer nachricht 

\section{Akka.Net Streams}

\subsection{GraphDSL}
\subsection{Overflow Strategien}


\section{Vergleich mit ähnlichen Systemen}
Kafka Streams EventHubs
\chapter{Konzept und Architektur}

\section{Gesamtarchitektur des Systems}

\section{Rollen im Cluster}
\subsection{Ingress}
\subsection{Backend}
\subsection{Coordinator}

\section{Proxy}

\subsection{Singleton-Proxy}
\subsection{Shard-Region-Proxy}

\section{Kommunikation über Verteiltes Pub/Sub}


\section{Alternative Architekturvarianten}

\chapter{Implementierung}

\section{Projektstruktur}

Die Projektstruktur ist so gewählt, dass der zentrale Anwendungsteil den vollständigen Code für das Kernmodul enthält. 
Innerhalb dieses Moduls werden die verschiedenen Rollen des Systems klar voneinander getrennt umgesetzt. 
Komponenten, die nicht direkt zum Kerngeschäft gehören, wie Infrastrukturkomponenten oder Tests, 
sind in eigenständige Projekte ausgelagert. Dadurch wird die Wartbarkeit verbessert und eine klare Trennung 
der Verantwortlichkeiten erreicht. Die Gesamtstruktur des Projekts ist in Abbildung~\ref{fig:project-structure} dargestellt.

\definecolor{folderbg}{RGB}{124,166,198}
\definecolor{folderborder}{RGB}{110,144,169}

\def\Size{4pt}
\tikzset{
  folder/.pic={
    \filldraw[draw=folderborder,top color=folderbg!50,bottom color=folderbg]
      (-1.05*\Size,0.2\Size+5pt) rectangle ++(.75*\Size,-0.2\Size-5pt);  
    \filldraw[draw=folderborder,top color=folderbg!50,bottom color=folderbg]
      (-1.15*\Size,-\Size) rectangle (1.15*\Size,\Size);
  }
}


\begin{forest}
  for tree={
    font=\ttfamily,
    grow'=0,
    child anchor=west,
    parent anchor=south,
    anchor=west,
    calign=first,
    inner xsep=7pt,
    edge path={
      \noexpand\path [draw, \forestoption{edge}]
      (!u.south west) +(7.5pt,0) |- (.child anchor) pic {folder} \forestoption{edge label};
    },
    before typesetting nodes={
      if n=1
        {insert before={[,phantom]}}
        {}
    },
    fit=band,
    before computing xy={l=15pt},
  }  
[BachelorAkkaNet
  [FormulaOneAkkaNet
    [Config]
    [Ingress]
    [Coordinator]
    [ShardRegion]
  ]
  [Infrastructure
    [General]
    [Http]
    [PubSub]
    [ShardRegion]
  ]
  [Client]
  [Tests]
]
\end{forest}

Die modulare Struktur des Projekts stellt sicher, dass die implementierten Rollen des Systems klar voneinander 
abgegrenzt sind. Jede Modulkomponente enthält ausschließlich die für ihre Rolle relevanten Aktorimplementierungen 
und Logik. Die einzige gemeinsam genutzte Komponente ist die zentrale Konfiguration, die im Ordner 
\textit{Config} abgelegt ist. Sie stellt sicher, dass alle Module konsistent in das .NET-Hostingmodell eingebunden 
werden können.

Das Projekt \textit{Infrastructure} umfasst wiederverwendbare Funktionen und Hilfskomponenten, die sowohl von den 
Hauptmodulen als auch von den Systemtests und dem Client genutzt werden. Im \textit{Client}-Projekt erfolgt die 
Darstellung und Auswertung der erfassten Daten, wodurch eine externe Visualisierung des Systemzustands möglich wird.

Für die Implementierung wurden zentrale Bibliotheken aus dem Akka.NET-Ökosystem verwendet. 
Dazu gehören Pakete für Clusterbildung, Sharding, Persistenz und Stream-Verarbeitung. 
Zusätzlich kommen Hosting-Erweiterungen zum Einsatz, welche die Konfiguration der Cluster- 
und Persistenzkomponenten über das .NET-Hostingmodell ermöglichen. Ergänzend werden Npgsql 
für den Datenbankzugriff und Serilog für das Logging eingesetzt. Tabelle~\ref{tab:packages} 
gibt einen Überblick über die relevanten Kernpakete.


\begin{table}[H]
\centering
\begin{tabular}{l l}
\textbf{Paket} & \textbf{Version} \\
\hline
Akka & 1.5.54 \\
Akka.Cluster & 1.5.54 \\
Akka.Cluster.Sharding & 1.5.54 \\
Akka.Hosting & 1.5.53 \\
Akka.Cluster.Hosting & 1.5.53 \\
Akka.Persistence & 1.5.54 \\
Akka.Persistence.Sql & 1.5.53 \\
Akka.Persistence.Hosting & 1.5.53 \\
Akka.Persistence.Sql.Hosting & 1.5.53 \\
Akka.Streams & 1.5.54 \\
Akka.Serialization.Hyperion & 1.5.54 \\
Npgsql & 9.0.4 \\
Serilog & 4.3.0 \\
\end{tabular}
\caption{Verwendete Kernbibliotheken und Versionen}
\label{tab:packages}
\end{table}

Die ausgewählten Pakete decken die zentralen technischen Anforderungen des Systems ab. Akka.Cluster und Akka.Cluster.Sharding 
ermöglichen eine verteilte und fehlertolerante Verarbeitung der eingehenden Daten. Akka.Persistence und die zugehörigen 
SQL-Erweiterungen stellen sicher, dass zustandsbehaftete Entitäten konsistent wiederhergestellt werden können. Akka.Streams 
wird für die kontrollierte Verarbeitung der Datenströme eingesetzt. Mit den Hosting-Erweiterungen wird eine einheitliche 
Konfiguration über das .NET-Hostingmodell ermöglicht, wodurch die Initialisierung des Clusters sowie der Persistence-Komponenten 
vereinfacht wird. Serilog und Npgsql ergänzen die Architektur durch Logging und Datenbankanbindung.


\section{Konfiguration und Initialisierung des Clusters}

\subsection{HOCON Konfiguration}

Die Konfiguration eines Akka.NET-Systems kann auf zwei unterschiedliche Arten erfolgen. 
Zum einen kann die HOCON-Konfiguration aus einer externen Datei eingelesen werden, die den vollständigen 
Konfigurationstext enthält. Zum anderen ermöglicht Akka.Hosting die Erstellung der Konfiguration direkt 
über Erweiterungsmethoden innerhalb des .NET-Hostingmodells.

Beide Ansätze wurden im Rahmen dieses Projekts eingesetzt. Für Konsolenanwendungen ist das Laden einer 
separaten HOCON-Datei zweckmäßig, da der Overhead des Hostingmodells dort nicht erforderlich ist. 
In Anwendungen, die auf dem .NET-Hostingmodell basieren, ist die programmgesteuerte Konfiguration über 
Akka.Hosting dagegen besonders vorteilhaft, da sie eine engere Integration in den Lebenszyklus des Hosts 
und eine klar strukturierte Initialisierung des Clusters ermöglicht.

\subsubsection*{HOCON File}

In klassischen Konsolenanwendungen ist es üblich, die Akka.NET-Konfiguration in einer separaten HOCON-Datei 
zu hinterlegen und beim Start der Anwendung einzulesen.

\begin{program}
\caption{Auszug aus der HOCON-Konfiguration des Clients}
\label{prog:hocon-client}
\begin{GenericCode}
akka {
    loglevel = "INFO"
    stdout-loglevel = "OFF"
    loggers = ["Akka.Logger.Serilog.SerilogLogger, Akka.Logger.Serilog"]
    
    actor {
      provider = cluster
      serializers {
        hyperion = "Akka.Serialization.HyperionSerializer, Akka.Serialization.Hyperion"
      }
      serialization-bindings {
        "System.Object" = hyperion
      }
    }
    
    remote.dot-netty.tcp { 
        hostname = "localhost"
        port = 0 
        maximum-frame-size = 2MiB
        send-buffer-size    = 2MiB
        receive-buffer-size = 2MiB
    
    }
    
    cluster {
      roles = ["api"]
      seed-nodes = [
        "akka.tcp://cluster-system@localhost:5000"
      ]
    }
}
\end{GenericCode}
\end{program}

Wie in Programm~\ref{prog:hocon-client} dargestellt, folgt die Konfigurationsdatei der HOCON-Syntax, 
die strukturell an JSON erinnert, jedoch flexibler und für Akka.NET optimiert ist. 
Jede Akka-Konfiguration beginnt mit dem Hauptelement \texttt{akka}, unter dem die zentralen 
Systemparameter definiert werden.

Die initialen Einträge betreffen die Konfiguration des Loggings. Anschließend wird über 
\texttt{actor.provider = cluster} festgelegt, dass dieser Prozess Teil eines verteilten Clusters ist. 
Die Serialisierung der Nachrichten wird mithilfe des Hyperion-Serialisierers konfiguriert, der eine 
effiziente binäre Datenrepräsentation ermöglicht und sich dadurch besonders für Clusterkommunikation eignet.

Der Block \texttt{remote.dot-netty.tcp} enthält die Netzwerkkonfiguration. 
Der Port ist auf \texttt{0} gesetzt, wodurch das System zur Laufzeit automatisch einen freien Port auswählt. 
Weitere Parameter wie \texttt{maximum-frame-size} oder die Puffergrößen steuern die maximale Nachrichtengröße 
und die Netzwerkeffizienz der Punkt-zu-Punkt-Kommunikation.

Im Abschnitt \texttt{cluster} werden die Rollen des Knotens sowie die \textit{seed nodes} festgelegt. 
Seed-Knoten dienen als initiale Kontaktpunkte beim Beitritt eines neuen Knotens zum Cluster. 
Nach der ersten Verbindung erfolgt die weitere Clusterentdeckung über das Gossip-Protokoll. 
Die Rollendefinition \texttt{api} legt fest, welche Aufgaben und Aktoren diesem Knoten im Clusterverbund zugeordnet sind.

\subsubsection*{Akka.Hosting}

\begin{program}
\caption{Auszug aus der Akka.Hosting-Konfiguration des Backend-Moduls}
\label{prog:hocon-hosting}
\begin{CsCode}

\end{CsCode}
\end{program}




Hier gehört rein:
HOCON-Konfiguration (Rollen, Ports, Seeds, Clustering)
ActorSystem-Initialisierung
Sharding-Konfiguration (EntityID, ShardID, Extractors)
Singleton-Konfiguration
DI-Setup via Akka.Hosting (wenn du Hosting nutzt)

\section{Eingangs-Service}

Datenabgriff von OpenF1
Datenabgriff von OpenF1
Normalisierung der Daten
Übergabe an Streams
Retry-Strategien
Fehlerbehandlung

\section{Datenbearbeitung mit Akka.Streams}

Aufbau des Stream-Graphs
Flow → Map → Filter → Buffer → Sink
Backpressure im konkreten System
Overflowstrategien (Verweis auf Kapitel 3)
Materialization

\section{Verarbeitung in der ShardRegion}

(„Datenbefüllung der ShardRegion mit Arbeiteraktoren“ ist ok, aber zu eng → besser allgemein formulieren)
Worker-/Driver-Actor
State-Management (telemetry state, sektoren, pace)
Event Verarbeitung
Persistence / Snapshots (falls vorhanden)
Lebenszyklus der Entitäten (Passivation, Shard-Migration)

\section{Nachrichtenverteilung im Cluster}

Distributed PubSub
Distributed PubSub:

Einrichtung des DistributedPubSub-Mediators
Subscription der Konsumenten
Publish von Events
Verwendung von Topics
Vorteile für lose Kopplung

\section{Cluster-Steuerung und Fehlertoleranz}
Hier erklärst du:
Coordinator-Singleton
Steuerung von ShardRegion-Verfügbarkeit
Handling von Node-Failures
Rebalancing-Verhalten (auf Implementationsebene)
Logs und Monitoring
Dieser Abschnitt zeigt deinem Professor, dass deine Architektur wirklich robust ist.

\section{Api oder Visualisierungsschicht}

Wenn du eine API oder UI hast, kannst du hier erklären:
Wie man auf Fahrer-Daten zugreift
Welche Endpunkte/Views es gibt
Wie Entity-Daten aggregiert dargestellt werden

\section{Zusammenführung der Komponenten}

Einfacher Abschnitt, um:
den Datenfluss von „OpenF1 → Streams → Sharding → PubSub“
End-to-End zu erklären
Das zeigt, dass du die Implementation im Ganzen verstehst.

\chapter{Tests und Analysen}

In diesem Kapitel werden die durchgeführten Tests und Analysen beschrieben, um einen detaillierten Einblick in die 
Funktionsweise, Zuverlässigkeit und Leistungsfähigkeit des entwickelten Systems zu geben. Darüber hinaus werden die 
erzielten Ergebnisse diskutiert und bewertet. Als Grundlage für die anschließenden Auswertungen werden zunächst die 
verwendete Teststrategie sowie die Testumgebung erläutert.

\section{Teststrategie und Testumgebung}

Zur Überprüfung der Funktionsfähigkeit und Robustheit des Systems wurden unterschiedliche Testarten eingesetzt. 
Diese Tests dienen sowohl der Validierung der funktionalen Anforderungen als auch der Analyse des Verhaltens der 
einzelnen Komponenten unter verschiedenen Betriebsbedingungen. Da das Gesamtsystem im Wesentlichen aus zwei 
zentralen Technologiebereichen besteht – der Akka.NET \texttt{ShardRegion} und der Datenverarbeitung über 
Akka.NET Streams – wurden die Testfälle entsprechend auf diese Bereiche abgestimmt.

Die durchgeführten Tests lassen sich in drei Kategorien einteilen:

\begin{itemize}
      \item \textbf{Funktionale Tests}: Überprüfung der korrekten Verarbeitung, Persistierung und Weiterleitung der Daten.
      \item \textbf{Failover- und Resilienztests}: Analyse des Systemverhaltens bei Node-Ausfällen, unerreichbaren 
          Clusterknoten sowie Neustarts der persistierten Entitäten.
      \item \textbf{Performance- und Durchsatztests}: Bewertung der Effizienz der implementierten Streamverarbeitungspipelines 
      sowie des Systemverhaltens unter Last.
\end{itemize}

Für die Tests auf Actor-Ebene wurde das Akka.NET \texttt{TestKit} eingesetzt. Dieses ermöglicht das Testen von 
Aktor-Interaktionen, Nachrichtenflüssen und Zustandsänderungen in einer kontrollierten, aber dennoch realitätsnahen 
Umgebung.

\section{Funktionale Tests}

Im Rahmen der funktionalen Tests wurde die korrekte Initialisierung und Zusammenarbeit der zentralen Cluster-Komponenten überprüft. 
Dabei wurde verifiziert, dass die ShardRegion erfolgreich erstellt wird und die Kommunikation zwischen mehreren Instanzen 
(Cluster-Knoten) erwartungsgemäß funktioniert.

Weiterhin wurde das Verhalten der Driver-Entities in verschiedenen Zuständen getestet. Insbesondere wurde geprüft, dass ein 
\texttt{DriverActorPersistent} vor der Initialisierung eingehende Nachrichten korrekt ablehnt und eine entsprechende Rückmeldung 
liefert. Nach erfolgreicher Initialisierung werden Telemetrie- und Zustandsupdates akzeptiert, persistiert und der interne Zustand 
konsistent aktualisiert.

Zusätzlich wurde validiert, dass verarbeitete Driver-Daten korrekt an nachgelagerte Komponenten weitergeleitet werden, 
insbesondere an die API sowie an die Monitoring-Komponente. Für Fehlerszenarien wurde geprüft, dass ungültige 
oder nicht zur jeweiligen Entität passende IDs zuverlässig erkannt werden und keine inkonsistenten Zustände entstehen.

\section{Verhalten bei Node-Ausfällen und Neuausrichtung}

Zur Bewertung der Fehlertoleranz des verteilten Systems wurden Szenarien mit ausfallenden und wieder hinzukommenden Cluster-Knoten 
untersucht. Dabei wurde insbesondere das Verhalten der ShardRegion sowie die Koordination durch den ClusterCoordinator analysiert.

Fällt ein Backend-Knoten aus, wird dies durch die Cluster-Mitgliedsüberwachung erkannt. Solange mindestens eine 
ShardRegion im Cluster verfügbar ist, wird die Verarbeitung weiterhin fortgesetzt und die betroffenen Shards werden 
auf verbleibende Knoten umverteilt (Rebalancing). Erst wenn keine ShardRegion mehr vorhanden ist, deaktiviert der 
ClusterCoordinator den Ingress-Dienst temporär, um eingehende Daten nicht an nicht verfügbare Backend-Komponenten 
weiterzuleiten.

Beim erneuten Hinzukommen eines Knotens werden die Shards neu gestartet und in den Cluster integriert. Die Persistenzmechanismen der 
\texttt{DriverActorPersistent}-Instanzen sorgen dabei dafür, dass der zuvor gespeicherte Zustand aus dem Journal beziehungsweise 
Snapshot wiederhergestellt wird. Dadurch bleibt der Zustand der Driver-Entities konsistent, und es gehen keine bereits verarbeiteten 
Telemetriedaten verloren.

Die Tests zeigen, dass das System Ausfälle einzelner Knoten korrekt erkennt, ein Rebalancing der Shards durchführt und nach 
Wiederherstellung der Knoten den persistierten Zustand zuverlässig lädt. Damit wird eine hohe Fehlertoleranz und Zustandskonsistenz 
im Clusterbetrieb gewährleistet.

\section{Leistungsanalyse und Bewertung der Stream-Strategien}

In diesem Abschnitt wird die Leistungsfähigkeit der implementierten Verarbeitungspipeline sowie der Einfluss 
unterschiedlicher Overflow-Strategien systematisch untersucht. Ziel ist es, das Verhalten der Pipeline unter 
Last zu analysieren und die Auswirkungen auf Durchsatz, Latenz und Nachrichtenverluste zu bewerten. 
Die Ergebnisse dienen dazu, die Eignung der gewählten Strategien für die Verarbeitung zeitkritischer 
Telemetriedaten zu beurteilen.

\subsection{Versuchsaufbau und Messgrößen}

Zur Untersuchung der Leistungsfähigkeit der implementierten Stream-Pipeline wurden experimentelle Messungen unter kontrollierten 
Bedingungen durchgeführt. Hierzu wurde ein synthetischer Datenstrom mit einer festen Anzahl von 4000 Elementen erzeugt und durch 
eine Akka.Streams-Pipeline geleitet. Die Einspeisung des Datenstroms erfolgte ohne explizite Ratenbegrenzung, sodass die Elemente 
unmittelbar erzeugt und in die Pipeline eingespeist wurden (Burst-Einspeisung). Die effektive Eingangsrate ergab sich somit aus 
der Verarbeitungskapazität der Pipeline, dem Backpressure-Mechanismus sowie der konfigurierten Puffergröße. 
Die Pipeline enthielt einen konfigurierbaren Puffer mit einer Größe von 50 Elementen sowie eine simulierte 
Verarbeitungsverzögerung von 2\,ms pro Element, um das Verhalten realer Telemetriedatenverarbeitung nachzubilden.

Als zentrale Messgrößen wurden der Durchsatz in Nachrichten pro Sekunde sowie die Latenzverteilung erfasst. Die Latenz wurde als 
Zeitdifferenz zwischen Erzeugung und Verarbeitung eines Elements bestimmt. Zur detaillierten Analyse wurden die Perzentile p50, 
p95 und p99 der Latenzverteilung berechnet. Zusätzlich wurde die Anzahl der tatsächlich verarbeiteten Elemente ermittelt, um 
mögliche Nachrichtenverluste in Abhängigkeit von der gewählten Overflow-Strategie zu quantifizieren.

\subsection{Vergleich der Overflow-Strategien}

Zur Bewertung der unterschiedlichen Overflow-Strategien wurden Durchsatz, Latenz und Nachrichtenverluste gemessen. 
Verglichen wurden die Strategien \texttt{Backpressure}, \texttt{DropNew}, \texttt{DropHead} und \texttt{DropTail}. 
Der Test verwendete einen Datenstrom von 4000 Elementen bei einer simulierten Verarbeitungszeit von 2\,ms pro Element 
und einer Puffergröße von 50 Elementen.

Der gemessene Durchsatz lag bei allen Strategien mit etwa 64 Nachrichten pro Sekunde auf einem ähnlichen Niveau. 
Dies zeigt, dass der Durchsatz primär durch die Verarbeitungsgeschwindigkeit der Pipeline und weniger durch die 
gewählte Overflow-Strategie bestimmt wird.

Deutliche Unterschiede zeigen sich bei der Anzahl verarbeiteter Elemente. Während Backpressure alle 4000 Elemente 
verlustfrei verarbeiten konnte, wurden bei den Drop-Strategien jeweils nur etwa 51 Elemente verarbeitet, da neue 
Elemente bei ausgelastetem Puffer verworfen werden.

Auch die Latenz unterscheidet sich deutlich: Backpressure weist mit einem Median von etwa 322\,ms und p95-Werten 
über 600\,ms höhere Latenzen auf, bedingt durch Warteschlangen im Puffer. Die Drop-Strategien erreichen dagegen 
niedrige Latenzen zwischen 11 und 21\,ms, da nur ein kleiner Teil der Elemente tatsächlich verarbeitet wird.

Insgesamt zeigt sich ein Zielkonflikt zwischen Vollständigkeit und Reaktionszeit. Backpressure ermöglicht eine 
verlustfreie Verarbeitung bei höheren Latenzen, während Drop-Strategien geringere Latenzen auf Kosten erheblicher 
Nachrichtenverluste bieten. Für die Verarbeitung kritischer Telemetriedaten erscheint daher Backpressure als die 
geeignetere Strategie.


\section{Diskussion der Ergebnisse}

Die durchgeführten Tests zeigen, dass die entwickelte Architektur die funktionalen und nicht-funktionalen Anforderungen 
im Wesentlichen erfüllt. Die funktionalen Tests bestätigen, dass die Interaktion zwischen Ingress-Service, 
Stream-Pipeline und Cluster-Sharding korrekt umgesetzt wurde und eingehende Telemetriedaten zuverlässig verarbeitet 
und weitergeleitet werden.

Die Failover-Tests verdeutlichen die hohe Fehlertoleranz des Systems. Durch die Verwendung von Akka.NET Cluster 
Sharding sowie persistenter Entities konnten Ausfälle einzelner Knoten erkannt und durch Rebalancing kompensiert 
werden. Gleichzeitig stellt die Wiederherstellung des Zustands aus Journal und Snapshots sicher, dass keine bereits 
verarbeiteten Daten verloren gehen. Dies bestätigt die Eignung der gewählten Architektur für verteilte, resiliente 
Stream-Verarbeitungssysteme.

Die Leistungsanalyse der Stream-Pipeline zeigt einen klaren Zielkonflikt zwischen Latenz und Vollständigkeit der 
Datenverarbeitung. Während Drop-basierte Overflow-Strategien niedrige Latenzen ermöglichen, führen sie zu erheblichen 
Nachrichtenverlusten. Die Backpressure-Strategie gewährleistet hingegen eine vollständige Verarbeitung aller Elemente, 
allerdings mit erhöhten Latenzen unter Last. Für den vorliegenden Anwendungsfall der Telemetriedatenverarbeitung ist 
diese Eigenschaft entscheidend, da Datenkonsistenz wichtiger ist als minimale Reaktionszeiten.

Es ist jedoch zu berücksichtigen, dass die Untersuchung der Overflow-Strategien in einem kontrollierten Testszenario 
mit synthetischen Daten, begrenzter Datenmenge und konstanter Verarbeitungszeit durchgeführt wurde. Die Ergebnisse 
liefern daher eine qualitative Einschätzung des Systemverhaltens unter Last, können jedoch nicht uneingeschränkt auf 
beliebig große Produktionsszenarien übertragen werden. Insbesondere können sich bei realen Telemetriedatenströmen mit 
variierender Last, Netzwerkverzögerungen und dynamischer Clustergröße abweichende Performancecharakteristiken ergeben.

Insgesamt bestätigen die Ergebnisse, dass die Kombination aus Akka.NET Cluster Sharding, persistierenden Entities 
und reaktiver Stream-Verarbeitung eine geeignete Grundlage für die skalierbare und fehlertolerante Verarbeitung 
kontinuierlicher Datenströme darstellt. Gleichzeitig zeigen die Tests, dass insbesondere unter hoher Last eine 
Abwägung zwischen Latenz und Datenvollständigkeit erforderlich ist, die je nach Anwendungsszenario unterschiedlich 
gewichtet werden muss.


\chapter{Zusammenfassung und Ausblick}

\section{Zusammenfassung der Ergebnisse}

Im Rahmen dieser Arbeit wurde ein verteiltes, aktorenbasiertes Stream-Verarbeitungssystem auf Basis von 
Akka.NET konzipiert und implementiert. Ziel war es, eingehende Telemetriedaten effizient, skalierbar und 
fehlertolerant zu verarbeiten sowie innerhalb eines Clusterverbunds zu speichern und weiterzuleiten.

Hierzu wurde eine Architektur entwickelt, die aus Ingress-, Backend- und Koordinator-Komponenten besteht. 
Der Eingangs-Service übernimmt den kontinuierlichen Empfang der Telemetriedaten über eine persistente 
Socket-Verbindung und speist diese in eine Akka.Streams-Pipeline ein. Durch die Verwendung einer 
\texttt{Source.Queue} mit Backpressure konnte ein kontrollierter Datenfluss zwischen externer Datenquelle 
und interner Verarbeitung realisiert werden.

Die Verarbeitung der Daten erfolgt innerhalb eines explizit definierten Stream-Graphs unter Verwendung der 
GraphDSL-API. Zur parallelen Verarbeitung wurden mehrere Worker-Aktoren eingesetzt, die über eine 
\texttt{Balance}-Stufe in den Stream integriert sind. Die Lastverteilung erfolgt dabei demand-basiert, wodurch 
eine dynamische Anpassung an die aktuelle Verarbeitungskapazität der Worker erreicht wird.

Zur Weiterleitung und Verarbeitung der Daten im Cluster wurde Akka.NET Cluster Sharding eingesetzt. Dadurch 
konnten zustandsbehaftete Entitäten verteilt und automatisch auf mehrere Knoten skaliert werden. Ergänzend 
wurde ein Acknowledgement-Mechanismus mittels \texttt{Sink.ActorRefWithAck} implementiert, um Backpressure 
auch auf Ebene der Aktor-Kommunikation zu unterstützen.

Die implementierte Architektur ermöglicht somit eine skalierbare, reaktive und fehlertolerante Verarbeitung 
kontinuierlicher Telemetriedatenströme innerhalb eines verteilten Aktorensystems.

Zur Überwachung der Kommunikation zwischen Ingress- und Backend-Komponenten wurde zusätzlich eine 
Koordinationsinstanz implementiert. Diese stellt sicher, dass der Ingress-Service eingehende Daten erst dann 
in das System einspeist, wenn mindestens eine empfangsbereite Backend-Komponente verfügbar ist. Hierzu werden 
Cluster-Ereignisse überwacht, insbesondere das Beitreten und Verlassen von Knoten sowie die Verfügbarkeit 
relevanter Aktoren. Auf diese Weise wird verhindert, dass Daten an nicht verfügbare Verarbeitungseinheiten 
gesendet werden, wodurch die Stabilität und Konsistenz des Systems erhöht wird.


\section{Bewertung der Zielerreichung}

Das primäre Ziel der Arbeit bestand darin, ein skalierbares und fehlertolerantes System zur Verarbeitung 
kontinuierlicher Telemetriedaten auf Basis von Akka.NET zu entwickeln. Dieses Ziel konnte weitgehend 
erreicht werden, da eine funktionale Architektur implementiert wurde, die eine verteilte Verarbeitung 
innerhalb eines Clusters ermöglicht.

Die Verwendung von Akka.Streams erwies sich als geeigneter Ansatz zur kontrollierten Verarbeitung der 
eingehenden Datenströme. Insbesondere der integrierte Backpressure-Mechanismus trug maßgeblich zur 
Stabilität des Systems bei, da eine Überlastung nachgelagerter Verarbeitungsschritte effektiv verhindert 
werden konnte. Durch den Einsatz mehrerer Worker-Aktoren in Kombination mit der \texttt{Balance}-Stufe 
konnte zudem eine dynamische Lastverteilung realisiert werden.

Auch die Integration von Cluster Sharding stellte einen wesentlichen Beitrag zur Zielerreichung dar, 
da dadurch zustandsbehaftete Entitäten automatisch auf mehrere Knoten verteilt und bei Bedarf neu 
ausbalanciert werden können. Dies erfüllt die Anforderungen an Skalierbarkeit und Fehlertoleranz 
verteilter Systeme.

Dennoch zeigten sich im Verlauf der Implementierung auch Einschränkungen. Die Komplexität der Konfiguration 
und des Debuggings verteilter Aktorensysteme erwies sich als hoch, insbesondere im Zusammenspiel von Streams, 
Clustering und Persistenz. Zudem hängt die tatsächliche Leistungsfähigkeit stark von der Anzahl der 
Worker-Aktoren sowie der verfügbaren Systemressourcen ab.

Insgesamt kann festgehalten werden, dass die gesetzten Ziele in Bezug auf Skalierbarkeit, Reaktivität und 
verteilte Verarbeitung erfolgreich umgesetzt wurden, wenngleich die praktische Komplexität solcher Systeme 
eine sorgfältige Konfiguration und Überwachung erfordert.

\section{Ausblick}

Zur weiteren Verbesserung des Systems bietet sich insbesondere der Einsatz von Cloud-Infrastrukturen an. 
Durch den Betrieb der Anwendung in einer containerisierten Umgebung könnte die Skalierbarkeit und Verfügbarkeit 
weiter erhöht werden. Container-Orchestrierungsplattformen wie Kubernetes ermöglichen dabei eine dynamische 
Anpassung der verfügbaren Ressourcen an die aktuelle Last und unterstützen somit den elastischen Betrieb 
verteilter Clusteranwendungen.

Darüber hinaus stellt die Kombination von Echtzeit- und historischen Daten einen vielversprechenden 
Erweiterungsansatz dar. Durch die parallele Verarbeitung beider Datenquellen könnten umfassendere Analysen 
sowie detailliertere Visualisierungen der Telemetriedaten realisiert werden. Dies würde es ermöglichen, 
aktuelle Rennsituationen in den Kontext vergangener Ereignisse zu setzen und weiterführende Auswertungen 
durchzuführen.

Ein weiterer zukünftiger Entwicklungsschritt betrifft die Integration von Logging- und Monitoring-Lösungen 
zur kontinuierlichen Überwachung der Systemleistung und -stabilität. Der Einsatz von Werkzeugen wie Prometheus 
zur Erfassung von Metriken sowie Grafana zur visuellen Aufbereitung dieser Daten könnte eine detaillierte 
Analyse des Systemverhaltens ermöglichen und die Wartbarkeit sowie den produktiven Betrieb des Systems 
nachhaltig verbessern.




\printbibliography

\end{document}
