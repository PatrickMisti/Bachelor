\chapter{Kurzfassung}

\begin{german}
Im Rahmen dieser Arbeit wurde ein Akka.NET-basiertes verteiltes System zur Verarbeitung von Formel-1-Telemetriedaten 
konzipiert, implementiert und evaluiert. Ziel des Use-Cases war es, kontinuierlich eintreffende Renndaten abzufangen, 
zu verarbeiten und in eine geeignete Datenstruktur zu transformieren, sodass diese resilient und fehlertolerant 
bereitgestellt werden können.

Ein zentraler Fokus lag dabei auf der Entwicklung einer Architektur, die eine robuste Verarbeitung auch bei auftretenden 
Fehlern oder Knotenausfällen gewährleistet. Durch den Einsatz des Aktorenmodells, asynchroner Nachrichtenkommunikation 
sowie Cluster-Sharding konnte ein System entworfen werden, das eingehende Datenströme zuverlässig verarbeitet und 
Zustandsänderungen konsistent verwaltet. Die implementierte Struktur erwies sich insbesondere im Hinblick auf 
Fehlertoleranz und Wiederherstellbarkeit nach Systemausfällen als geeignet.

Während der Umsetzung zeigte sich, dass die Konfiguration und Einrichtung eines verteilten Akka.NET-Clusters mit 
erheblichem Aufwand verbunden ist. Insbesondere die korrekte Abstimmung der Cluster-, Sharding- und Persistenzkomponenten 
erforderte eine intensive Einarbeitung, was teilweise auf die vergleichsweise geringere Verbreitung von Akka.NET im 
Vergleich zu anderen Streaming- oder Microservice-Technologien zurückzuführen ist.

Die Evaluation verdeutlicht jedoch, dass sich Akka.NET besonders für Szenarien eignet, in denen häufige Zustandsänderungen 
in Echtzeit verarbeitet werden müssen. Die Kombination aus asynchroner Aktorenkommunikation, integrierter Fehlertoleranz 
und automatischer Wiederherstellung von Zuständen ermöglicht eine stabile und resiliente Datenverarbeitung auch unter 
hoher Last oder bei partiellen Systemausfällen.

Ein weiterer Vorteil besteht in der Unterstützung verteilter Ausführung über Remote-Kommunikation, wodurch Ressourcen 
mehrerer Rechner effizient genutzt werden können. Dies eröffnet zusätzliche Möglichkeiten zur horizontalen Skalierung 
und zur besseren Auslastung verfügbarer Systemressourcen.
% ...
\end{german}

