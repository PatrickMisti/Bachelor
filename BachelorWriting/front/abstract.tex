\chapter{Abstract}


\begin{english} %switch to English language rules
In this thesis, Akka.NET was evaluated based on a concrete use case involving the processing
of Formula 1 telemetry data. The goal of the use case was to capture continuously incoming
race data, process it, and transform it into a suitable data structure that can be provided
in a resilient and fault-tolerant manner.

A central focus was the design of an architecture capable of maintaining reliable data
processing even in the presence of errors or node failures. By leveraging the actor model,
asynchronous message passing, and cluster sharding, a distributed system was implemented
that consistently handles streaming data and manages state changes in a robust way.
The resulting structure proved to be particularly effective with respect to fault tolerance
and state recovery after system crashes.

During the implementation, it became evident that configuring and operating a distributed
Akka.NET cluster requires considerable effort. In particular, the correct setup and tuning
of cluster, sharding, and persistence components demanded extensive configuration work.
This complexity can partly be attributed to the comparatively lower popularity of Akka.NET
compared to other streaming or microservice technologies, which results in fewer practical
resources and examples.

However, the evaluation shows that Akka.NET is especially well suited for scenarios in which
frequent state updates must be processed in real time. The combination of asynchronous actor
communication, built-in fault tolerance, and automatic state recovery enables stable and
resilient data processing even under high load or partial system failures.

Another important advantage is the support for remote and distributed execution. This allows
the system to utilize resources across multiple machines, enabling horizontal scalability and
a more efficient use of available computational resources.
% ...
\end{english}

