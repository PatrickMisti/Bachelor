\chapter{Zusammenfassung und Ausblick}

\section{Zusammenfassung der Ergebnisse}

Im Rahmen dieser Arbeit wurde ein verteiltes, aktorenbasiertes Stream-Verarbeitungssystem auf Basis von 
Akka.NET konzipiert und implementiert. Ziel war es, eingehende Telemetriedaten effizient, skalierbar und 
fehlertolerant zu verarbeiten sowie innerhalb eines Clusterverbunds zu speichern und weiterzuleiten.

Hierzu wurde eine Architektur entwickelt, die aus Ingress-, Backend- und Koordinator-Komponenten besteht. 
Der Eingangs-Service übernimmt den kontinuierlichen Empfang der Telemetriedaten über eine persistente 
Socket-Verbindung und speist diese in eine Akka.Streams-Pipeline ein. Durch die Verwendung einer 
\texttt{Source.Queue} mit Backpressure konnte ein kontrollierter Datenfluss zwischen externer Datenquelle 
und interner Verarbeitung realisiert werden.

Die Verarbeitung der Daten erfolgt innerhalb eines explizit definierten Stream-Graphs unter Verwendung der 
GraphDSL-API. Zur parallelen Verarbeitung wurden mehrere Worker-Aktoren eingesetzt, die über eine 
\texttt{Balance}-Stufe in den Stream integriert sind. Die Lastverteilung erfolgt dabei demand-basiert, wodurch 
eine dynamische Anpassung an die aktuelle Verarbeitungskapazität der Worker erreicht wird.

Zur Weiterleitung und Verarbeitung der Daten im Cluster wurde Akka.NET Cluster Sharding eingesetzt. Dadurch 
konnten zustandsbehaftete Entitäten verteilt und automatisch auf mehrere Knoten skaliert werden. Ergänzend 
wurde ein Acknowledgement-Mechanismus mittels \texttt{Sink.ActorRefWithAck} implementiert, um Backpressure 
auch auf Ebene der Aktor-Kommunikation zu unterstützen.

Die implementierte Architektur ermöglicht somit eine skalierbare, reaktive und fehlertolerante Verarbeitung 
kontinuierlicher Telemetriedatenströme innerhalb eines verteilten Aktorensystems.

Zur Überwachung der Kommunikation zwischen Ingress- und Backend-Komponenten wurde zusätzlich eine 
Koordinationsinstanz implementiert. Diese stellt sicher, dass der Ingress-Service eingehende Daten erst dann 
in das System einspeist, wenn mindestens eine empfangsbereite Backend-Komponente verfügbar ist. Hierzu werden 
Cluster-Ereignisse überwacht, insbesondere das Beitreten und Verlassen von Knoten sowie die Verfügbarkeit 
relevanter Aktoren. Auf diese Weise wird verhindert, dass Daten an nicht verfügbare Verarbeitungseinheiten 
gesendet werden, wodurch die Stabilität und Konsistenz des Systems erhöht wird.


\section{Bewertung der Zielerreichung}

Das primäre Ziel der Arbeit bestand darin, ein skalierbares und fehlertolerantes System zur Verarbeitung 
kontinuierlicher Telemetriedaten auf Basis von Akka.NET zu entwickeln. Dieses Ziel konnte weitgehend 
erreicht werden, da eine funktionale Architektur implementiert wurde, die eine verteilte Verarbeitung 
innerhalb eines Clusters ermöglicht.

Die Verwendung von Akka.Streams erwies sich als geeigneter Ansatz zur kontrollierten Verarbeitung der 
eingehenden Datenströme. Insbesondere der integrierte Backpressure-Mechanismus trug maßgeblich zur 
Stabilität des Systems bei, da eine Überlastung nachgelagerter Verarbeitungsschritte effektiv verhindert 
werden konnte. Durch den Einsatz mehrerer Worker-Aktoren in Kombination mit der \texttt{Balance}-Stufe 
konnte zudem eine dynamische Lastverteilung realisiert werden.

Auch die Integration von Cluster Sharding stellte einen wesentlichen Beitrag zur Zielerreichung dar, 
da dadurch zustandsbehaftete Entitäten automatisch auf mehrere Knoten verteilt und bei Bedarf neu 
ausbalanciert werden können. Dies erfüllt die Anforderungen an Skalierbarkeit und Fehlertoleranz 
verteilter Systeme.

Dennoch zeigten sich im Verlauf der Implementierung auch Einschränkungen. Die Komplexität der Konfiguration 
und des Debuggings verteilter Aktorensysteme erwies sich als hoch, insbesondere im Zusammenspiel von Streams, 
Clustering und Persistenz. Zudem hängt die tatsächliche Leistungsfähigkeit stark von der Anzahl der 
Worker-Aktoren sowie der verfügbaren Systemressourcen ab.

Insgesamt kann festgehalten werden, dass die gesetzten Ziele in Bezug auf Skalierbarkeit, Reaktivität und 
verteilte Verarbeitung erfolgreich umgesetzt wurden, wenngleich die praktische Komplexität solcher Systeme 
eine sorgfältige Konfiguration und Überwachung erfordert.

\section{Ausblick}

Zur weiteren Verbesserung des Systems bietet sich insbesondere der Einsatz von Cloud-Infrastrukturen an. 
Durch den Betrieb der Anwendung in einer containerisierten Umgebung könnte die Skalierbarkeit und Verfügbarkeit 
weiter erhöht werden. Container-Orchestrierungsplattformen wie Kubernetes ermöglichen dabei eine dynamische 
Anpassung der verfügbaren Ressourcen an die aktuelle Last und unterstützen somit den elastischen Betrieb 
verteilter Clusteranwendungen.

Darüber hinaus stellt die Kombination von Echtzeit- und historischen Daten einen vielversprechenden 
Erweiterungsansatz dar. Durch die parallele Verarbeitung beider Datenquellen könnten umfassendere Analysen 
sowie detailliertere Visualisierungen der Telemetriedaten realisiert werden. Dies würde es ermöglichen, 
aktuelle Rennsituationen in den Kontext vergangener Ereignisse zu setzen und weiterführende Auswertungen 
durchzuführen.

Ein weiterer zukünftiger Entwicklungsschritt betrifft die Integration von Logging- und Monitoring-Lösungen 
zur kontinuierlichen Überwachung der Systemleistung und -stabilität. Der Einsatz von Werkzeugen wie Prometheus 
zur Erfassung von Metriken sowie Grafana zur visuellen Aufbereitung dieser Daten könnte eine detaillierte 
Analyse des Systemverhaltens ermöglichen und die Wartbarkeit sowie den produktiven Betrieb des Systems 
nachhaltig verbessern.

