\chapter{Einleitung}

Moderne Softwaresysteme stehen zunehmend vor der Herausforderung, kontinuierlich anfallende Datenströme 
performant, skalierbar und fehlertolerant zu verarbeiten \parencite{boner2014reactivemanifesto,kuhn2017reactivedesignpatterns}. 
Insbesondere in datenintensiven Anwendungsdomänen wie der Analyse von Telemetriedaten entstehen hochfrequente Datenströme, 
die in Echtzeit erfasst, verarbeitet und bereitgestellt werden müssen. Klassische monolithische Architekturen 
stoßen hierbei schnell an ihre Grenzen, da sie nur begrenzt skalierbar sind und eine geringe Fehlertoleranz 
aufweisen \parencite{kuhn2017reactivedesignpatterns}.

Reaktive Architekturen auf Basis des Aktorenmodells bieten einen vielversprechenden Ansatz zur Bewältigung 
dieser Anforderungen \parencite{boner2014reactivemanifesto}. Durch lose Kopplung, natürliche Parallelisierung sowie eingebaute Fehlertoleranz 
eignen sie sich besonders für die Verarbeitung kontinuierlicher Datenströme. Frameworks wie Akka.NET ermöglichen 
die Entwicklung verteilter Aktorensysteme, während Akka.Streams reaktive Datenpipelines mit integrierten 
Backpressure-Mechanismen bereitstellt \parencite{kuhn2017reactivedesignpatterns}. In Kombination mit Cluster Sharding können zustandsbehaftete Entitäten 
über mehrere Knoten verteilt und dynamisch skaliert werden \parencite{kuhn2017reactivedesignpatterns}.

\section{Motivation}

Die Motivation dieser Arbeit ergibt sich aus der Notwendigkeit, kontinuierliche Telemetriedaten effizient und 
skalierbar zu verarbeiten. Im Kontext des Motorsports fallen während eines Rennens große Mengen an Daten an, 
beispielsweise Positions-, Geschwindigkeits- und Zustandsinformationen einzelner Fahrzeuge. Diese Daten müssen 
in Echtzeit verarbeitet werden, um aktuelle Rennsituationen analysieren und visualisieren zu können.

Gleichzeitig müssen solche Systeme auch unter hoher Last stabil bleiben und Ausfälle einzelner Komponenten 
kompensieren können. Verteilte Aktorensysteme in Kombination mit streambasierter Datenverarbeitung stellen 
hierfür einen geeigneten Ansatz dar.

\section{Aufgabenstellung}

Ziel dieser Arbeit ist die Konzeption und Implementierung eines verteilten, aktorenbasierten 
Stream-Verarbeitungssystems zur Verarbeitung kontinuierlicher Telemetriedaten. Dabei soll ein System entwickelt 
werden, das eingehende Datenströme zuverlässig empfängt, verarbeitet und innerhalb eines Clusterverbunds verteilt bereitstellt.

Die Architektur basiert auf dem Akka.NET-Framework und integriert zentrale Konzepte wie Cluster Sharding, reaktive 
Stream-Verarbeitung mit Backpressure sowie eine koordinierte Kommunikation zwischen den Systemkomponenten.

\section{Ziel der Arbeit}

Das Ziel der Arbeit besteht in der Entwicklung und prototypischen Umsetzung einer skalierbaren und fehlertoleranten 
Architektur zur Echtzeitverarbeitung von Telemetriedaten. Es soll gezeigt werden, wie durch den Einsatz von Akka.NET, 
Akka.Streams und Cluster Sharding eine verteilte Verarbeitung kontinuierlicher Datenströme realisiert werden kann.

Darüber hinaus soll die entwickelte Architektur hinsichtlich ihrer Eignung für reaktive Systeme bewertet und ihre 
Erweiterbarkeit für zukünftige Anwendungen aufgezeigt werden.

\section{Vorgehensweise}

Zu Beginn werden die theoretischen Grundlagen zu reaktiven Systemen, dem Aktorenmodell sowie der Stream-Verarbeitung 
erläutert. Darauf aufbauend erfolgt die Konzeption der Systemarchitektur, in der die Rollen der Cluster-Komponenten 
definiert und deren Zusammenspiel beschrieben werden. Anschließend wird die Implementierung des Systems detailliert 
vorgestellt. Abschließend werden Tests und Analysen durchgeführt, um das Verhalten des Systems hinsichtlich Skalierbarkeit, 
Fehlertoleranz und Leistungsfähigkeit zu evaluieren.
